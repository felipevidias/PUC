% Nome do capítulo
\chapter{Primeiro capítulo de exemplo}
% Label para referenciar
\label{cap1}

% Diminuir espaçamento entre título e texto
\vspace{-1.9cm}

% Texto do capítulo

  A seguir serão apresentados alguns comandos do LaTex usados comumente para formatar textos de dissertação baseados
  na normalização da PUC (2011).

  Para as citações a norma estabelece duas formas de apresentação. A primeira delas é empregada quando a
  citação aparece no final de um parágrafo. Neste caso, o comando cite é usado para formatar a citação em caixa alta,
  como é mostrado no exemplo a seguir. \cite{Duato:2002}.

  Outra forma de apresentação da citação é a que ocorre no decorrer do texto, essa situação é exemplificada na próxima frase.
  Conforme \citeonline{Bjerregaard:2006}, o estudo mencionado revela progressos no desempenho dos processadores. 
  Para a formatação da citação em caixa baixa deve ser usado o comando citeonline.

  Nas citações que aparecem mais de uma referência as mesmas devem ser separadas por vírgulas, como
  neste exemplo. \cite{Keyes:2008, Zhao:2008, Ganguly:2011}. Se houver necessidade de especificar a página ou que foi
  realizada uma tradução do texto deve ser feito da seguinte maneira. \cite[p.~2, tradução nossa]{Sasaki:2009}.
  A citação direta deve ser feita de forma semelhante. ``[...] A carga de trabalho de um sistema pode 
  ser definida como o conjunto de todas as informações de entrada.''~\cite[p. 160]{Menasce:2002}.

  O arquivo dissertacao.bib mostra exemplos de representação para vários tipos de referências (artigos de conferências, 
  periódicos, relatórios, livros, dentre outros). Cada um desses tipos requer uma forma diferente de representação para 
  que a referência seja formatada conforme as exigências da normalização.

\section{Primeira seção}
\label{secao1}

  Para gerar a lista de siglas automaticamente deve ser usado o pacote \textit{acronym}. Para tanto, toda vez que uma sigla for mencionada no texto
  deve ser usado o comando ac\{sigla\}. Dessa forma, se for a primeira ocorrência da sigla a mesma será escrita por extenso
  conforme descrição feita no arquivo lista-siglas.tex. Caso contrário, somente a sigla será mostrada. Ex

\subsection{Primeira subseção}
  
  As enumerações devem ser geradas usando o pacote \textit{compactitem}. Cada item deve terminar com um ponto final.
  Abaixo um exemplo de enumeração é apresentado:

    \begin{compactitem}
      \item[a)] Coletar e analisar.
      \item[b)] Configurar e simular.
      \item[c)] Definir a metodologia.
      \item[d)] Avaliar o desempenho.
      \item[e)] Analisar e avaliar características.
    \end{compactitem}

\section{Segunda seção}

  Para referenciar um capítulo, seção ou subseção basta definir um label para o mesmo e usar o comando ref para referênciá-lo
  no texto. Exemplo: Como pode ser visto no Capítulo \ref{cap1} ou na Seção \ref{secao1}.